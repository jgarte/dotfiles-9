\documentclass{article}

\usepackage{noweb}
\usepackage{minted}

\begin{document}

This is a sample file for the {\tt{}noweb-minted} filter. The filter uses
the pygments library to guess the correct lexer based on the extension
of the file given in a code chunk. If no lexer is detected, the
``text'' lexer is used, which performs no pretty printing. For now,
the filter only puts the {\tt mathescape} option into the minted
environment. I think the right way to do it is to externally set the
correct parameters using {\tt \textbackslash{setminted}} and related
commands. Examples:

\begin{itemize}
\item C

\nwfilename{sample.nw}\moddef{hello.c}\endmoddef
\begin{minted}[mathescape]{C}
#include <stdio.h>

int main() {
  printf("Hello, World!\n");
  return 0;    
}

\end{minted}
\nwbegindocs{2}\nwdocspar

\item Haskell

\nwenddocs{}\moddef{hello.hs}\endmoddef
\begin{minted}[mathescape]{Haskell}
main :: IO ()
main =  putStrLn "Hello, world\n"

\end{minted}
\nwbegindocs{4}\nwdocspar

\item Coq

\nwenddocs{}\moddef{hello.v}\endmoddef
\begin{minted}[mathescape]{Coq}
Require Import CoqIO.IO.
Require Import ExtLib.Programming.Show.

Import ShowNotation.
Local Open Scope show_scope.

Definition main : IO unit :=
  runShow (M := ShowScheme_Std StdOut) 
          (show_exact "Hello world!" << Char.chr_newline).

\end{minted}
\nwbegindocs{6}\nwdocspar

\item No lexer

\nwenddocs{}\moddef{hello.foobar}\endmoddef
\begin{minted}[mathescape]{text}
int main () {
  printFoo(Bar);
}

\end{minted}
\nwbegindocs{8}\nwdocspar

\end{itemize}

\end{document}
\nwenddocs{}
